  %------------------------------------------------------------------------------
% CV in Latex
% Author : Charles Rambo
% Based off of: https://github.com/sb2nov/resume and Jake's Resume on Overleaf
% Most recently updated version may be found at https://github.com/fizixmastr 
% License : MIT
%------------------------------------------------------------------------------

\documentclass[A4,11pt]{article}
%\documentclass[letterpaper,11pt]{article} %For use in US
\usepackage{latexsym}
\usepackage[empty]{fullpage}
\usepackage{titlesec}
\usepackage{marvosym}
\usepackage[usenames,dvipsnames]{color}
\usepackage{verbatim}
\usepackage{enumitem}
\usepackage[hidelinks]{hyperref}
\usepackage[english]{babel}
\usepackage{tabularx}
\usepackage{tikz}
\input{glyphtounicode}

\begin{comment}
I am by no means a professional when it comes to the CV's/resumes, I have
received various trainings on how to write a CV and resume from my high 
school, as well as the Austin College and University of Eastern Finland's
career counseling departments. As I intend to share my CV as a template, I 
feel that it is my responsibility to provide explanations of my work.
\end{comment}


%-----FONT OPTIONS-------------------------------------------------------------
\begin{comment}
The font of the document will impact not just how readable it is, but how it is
perceived. In the "The Craft of Scientific Writing" by Michael Alley, shares a
common fonts for publication as well as their use. I have chosen to use
Palatino for its legibility, some others are given below. There is far too much
about typography to discus here. Note: serif fonts have short projecting
strokes, sans-serif fonts are sans (without) these strokes.
\end{comment}


% serif
 \usepackage{palatino}
% \usepackage{times} %This is the default as well
% \usepackage{charter}

% sans-serif
% \usepackage{helvet}
% \usepackage[sfdefault]{noto-sans}
% \usepackage[default]{sourcesanspro}

%-----PAGE SETUP---------------------------------------------------------------

% Adjust margins
\addtolength{\oddsidemargin}{-1cm}
\addtolength{\evensidemargin}{-1cm}
\addtolength{\textwidth}{2cm}
\addtolength{\topmargin}{-1cm}
\addtolength{\textheight}{2cm}

% Margins for US Letter size
%\addtolength{\oddsidemargin}{-0.5in}
%\addtolength{\evensidemargin}{-0.5in}
%\addtolength{\textwidth}{1in}
%\addtolength{\topmargin}{-.5in}
%\addtolength{\textheight}{1.0in}

\urlstyle{same}

\raggedbottom
\raggedright
\setlength{\tabcolsep}{0cm}

% Sections formatting
\titleformat{\section}{
  \vspace{-4pt}\scshape\raggedright\large
}{}{0em}{}[\color{black}\titlerule \vspace{-5pt}]

% Ensure that .pdf is machine readable/ATS parsable
\pdfgentounicode=1

%-----CUSTOM COMMANDS FOR FORMATTING SECTIONS----------------------------------
\newcommand{\CVItem}[1]{
  \item\small{
    {#1 \vspace{-2pt}}
  }
}

\newcommand{\CVSubheading}[4]{
  \vspace{-2pt}\item
    \begin{tabular*}{0.97\textwidth}[t]{l@{\extracolsep{\fill}}r}
      \textbf{#1} & #2 \\
      \small#3 & \small #4 \\
    \end{tabular*}\vspace{-7pt}
}

\newcommand{\CVSubSubheading}[2]{
    \item
    \begin{tabular*}{0.97\textwidth}{l@{\extracolsep{\fill}}r}
      \text{\small#1} & \text{\small #2} \\
    \end{tabular*}\vspace{-7pt}
}

\newcommand{\CVSubItem}[1]{\CVItem{#1}\vspace{-4pt}}

\renewcommand\labelitemii{$\vcenter{\hbox{\tiny$\bullet$}}$}

\newcommand{\CVSubHeadingListStart}{\begin{itemize}[leftmargin=0.5cm, label={}]}
% \newcommand{\resumeSubHeadingListStart}{\begin{itemize}[leftmargin=0.15in, label={}]} % Uncomment for US
\newcommand{\CVSubHeadingListEnd}{\end{itemize}}
\newcommand{\CVItemListStart}{\begin{itemize}}
\newcommand{\CVItemListEnd}{\end{itemize}\vspace{-5pt}}

%------------------------------------------------------------------------------
% CV STARTS HERE  %
%------------------------------------------------------------------------------
\begin{document}

%-----HEADING------------------------------------------------------------------
\begin{comment}
In Europe it is common to include a picture of ones self in the CV. Select
which heading appropriate for the document you are creating.
\end{comment}


\begin{center}
   \textbf{\Huge \scshape Yurev Alexander} \\ \vspace{1pt} %\scshape sets small capital letters, remove if desired
   \small +7 919-545-5574 $|$ 
   \href{mailto:alexander.yurev@internet.ru}{\underline{alexander.yurev@internet.ru}} $|$\\
   % Be sure to use a professional *personal* email address
   \href{https://linkedin.com/in/yurevalexander}{\underline{linkedin.com/in/yurevalexander}} $|$
   % you should adjust you linked in profile name to be professional and recognizable
   \href{https://github.com/Sapfir0}{\underline{github.com/Sapfir0}}

   \href{https://habr.com/ru/users/alexanderyurev/}{\underline{habr.com/ru/users/alexanderyurev}}
\end{center}



\begin{comment}
This CV was written for specifically for positions I was applying for in
academia, and then modified to be a template.

A standard CV is about two pages long where as a resume in the US is one page.
sections can be added and removed here with this in mind. In my experience, 
education, and applicable work experience and skills are the most import things
to include on a resume. For a CV the Europass CV suggests the categories: Work
Experience, Education and Training, Language Skills, Digital Skills,
Communication and Interpersonal Skills, Conferences and Seminars, Creative Works
Driver's License, Hobbies and Interests, Honors and Awards, Management and
Leadership Skills, Networks and Memberships, Organizational Skills, Projects,
Publications, Recommendations, Social and Political Activities, Volunteering.

Your goal is to convey a who, what , when, where, why for every item you share. 
The who is obviously you, but I believe the rest should be done in that order.
For example below. An employer cares most about the degree held and typically 
less about the institution or where it is located (This is still good 
information though). Whatever order you choose be consistent throughout.
\end{comment}

%-----EDUCATION----------------------------------------------------------------
\section{Education}
  \CVSubHeadingListStart
%    \CVSubheading % Example
%      {Degree Achieved}{Years of Study}
%      {Institution of Study}{Where it is located}
    \CVSubheading
      {{Bachelor of Computer Science $|$ \emph{\small{Major: Programming, Minor: Management}}}}{Sep. 2017 -- July 2021}
      {Volgograd State Technical University}{Volgograd, Russia}
  \CVSubHeadingListEnd

%-----WORK EXPERIENCE----------------------------------------------------------
\begin{comment}
try to briefly explain what you did and why it is relevant to the position you
are seeking
\end{comment}

\section{Work Experience}
  \CVSubHeadingListStart
%    \CVSubheading %Example
%      {What you did}{When you worked there}
%      {Who you worked for}{Where they are located}
%      \CVItemListStart
%        \CVItem{Why it is important to this employer}
%      \CVItemListEnd
    \CVSubheading
      {Junior Frontend Developer}{November 2020 -- Present}
      {Singularis Lab LLC.}{Volgograd, Russia}
      \CVItemListStart
        \CVItem{Developed a new iteration of the Volma Drivers application}
        \CVItem{Performed tasks to deleloping the client side of IDM Clinic}
      \CVItemListEnd
    \CVSubheading
      {Frontend Intern}{July 2020 -- October 2020}
      {Singularis Lab LLC.}{Volgograd, Russia}
      \CVItemListStart
        \CVItem{Developed client side application of manual testing system on TypeScript, React}
    \CVItemListEnd
    \CVSubheading
      {Intern Programmer}{January 2020 -- April 2020}
      {Intervolga}{Volgograd, Russia}
      \CVItemListStart
        \CVItem{Delelop web application UI and backend logic on PHP}
      \CVItemListEnd
  \CVSubHeadingListEnd

%-----PROJECTS AND RESEARCH----------------------------------------------------
\begin{comment}
Ideally the title of the work should speak for what it is. However if you feel
like you should explain more about why the project is applicable to this job,
use item list as is shown in the work experience section.
\end{comment}

\section{Projects and Research}
  \CVSubHeadingListStart
%    \CVSubheading
%      {Title of Work}{When it was done}
%      {Institution you worked with}{unused}
    \CVSubheading
      {{Premier-eye} $|$ \emph{\small{Python, React, Typescript, Flask}}}{2020-2021}
      {Recognition program with an API and web interface. Recognizes cars and people from pictures. \\
      Separate components were written, packed in docker containers.}{}
    \CVSubheading
      {{Liquid Cloud} $|$ \emph{\small{Elixir, TypeScript, React}}}{2021}
      {Small home cloud on functional language.}{}
    \CVSubheading
      {{Metida} $|$ \emph{\small{Node.js, TypeScript, React}}}{2019-2020}
      {Team project. I was mainly engaged in frontend. \\
      The project is the server and client parts of the blog platform with articles.}{}
    \CVSubheading
      {{Arduino meteostation} $|$ \emph{\small{C++, JavaScript}}}{2019}
      {Implemented work with OpenWeatherMap API. \\
      There is a measurement of the main indicators inside the room. \\
      The data is presented on the LCD screen.}{}
  \CVSubHeadingListEnd


%-----CONFERENCES AND PRESENTATIONS--------------------------------------------
\begin{comment}
Again the title should have already been enough, but if it is necessary to add
descriptions maintain the consistency from prior sections
\end{comment}

\section{Conferences and Presentations}
  \CVSubHeadingListStart
%    \CVSubheading % Example
%      {Work Presented}{When}
%      {Occasion}{}
    \CVSubheading
      {Winter school URFU "Software engineering and radio engieering"}{Spring 2020}
      {During the five days of school, I’m learned how improve the soft-skills. }{Sochi, Russia}
    \CVSubheading
      { Project “Campus” SKB KONTUR }{Fall 2019}
      { As part of a four-day field school, he learned the basics of designing in C\#, \\
      got the skills in using Git, DI containers and the skills in writing clean code. }{ Yekaterinburg, Russia }
  \CVSubHeadingListEnd

%-----HONORS AND AWARDS--------------------------------------------------------
\section{Honors and Awards}
  \CVSubHeadingListStart
%    \CVSubheading %Example
%      {What}{When}
%      {Short Description}{}
    \CVSubheading
      {Participation in the final of the olympiad "I am professional"}{Winter 2020}
      {Participation on "Software engieering" track}{Moscow, Russia}
  \CVSubHeadingListEnd

%-----SKILLS-------------------------------------------------------------------
\begin{comment}
This section is compressed from the various skills sections that Euro CV
recommends.
\end{comment}

\section{Skills}
 \begin{itemize}[leftmargin=0.5cm, label={}]
    \small{\item{
     \textbf{Languages}{: English (B2), Russian (Native) } \\
     \textbf{Programming}{: TypeScript, JavaScript, Python, C\# } \\
     \textbf{Frontend}{: React, Redux, MobX, Redux-saga } \\
     \textbf{Technologies}{: Git, Docker } \\
    }}
 \end{itemize}
    
%------------------------------------------------------------------------------
\end{document}